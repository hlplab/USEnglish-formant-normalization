\usepackage{tipa}
\usepackage{tikz}
\usetikzlibrary{bayesnet}
% rotating must be loaded before tablefootnote
% \usepackage{rotating}
\usepackage{multirow}
\usepackage{colortbl}
\usepackage{tablefootnote}
\usepackage{amssymb}
\usepackage[caption=false]{subfig}
\usepackage{setspace}
% For references to external document
\usepackage{xr}
\externaldocument{manuscript-with-SI}

% Usually, we would use Latex package fontspec with Latex command \ipatext and Doulos SIL, 
% in combination with the linguisticsdown library in R to deal with unicode symbols (incl. 
% IPA). But JASA requires 
% that we use Latex package tipa with commmand textipa instead
% \usepackage{fontspec}
% 
% %% Special font for IPA
% %% Make sure "Doulos SIL" is installed on your computer
% %% For other typefaces supporting IPA symbols, see
% %% https://en.wikipedia.org/wiki/International_Phonetic_Alphabet#Typefaces
% \newfontfamily\ipa{Doulos SIL} % Font for IPA symbols
% \DeclareTextFontCommand{\ipatext}{\ipa}
% Text command for tipa
\DeclareTextFontCommand{\ipatext}{\textipa}

%%%         Section for CJK Characters                   %%%
%%%   You may want to uncomment the code below if        %%%
%%%   you're writing this document with CJK characters   %%%

%\usepackage{xeCJK}  % Uncomment for using CJK characters
%% Set main font for CJK characters
%% Make sure your system has the font set
%\setCJKmainfont[
%	BoldFont={HanWangHeiHeavy}  % Set font for CJK boldface
%    ]{標楷體}    % Set font for normal CJK
%% Some Traditional Chinese fonts: AR PL KaitiM Big5, PingFang TC, Noto Sans CJK TC
%\XeTeXlinebreaklocale "zh"
%\XeTeXlinebreakskip = 0pt plus 1pt

%% Added by Florian to make biblatex and its \refsection work since pandoc does not parse 
%% content within latex environments. \refsection in turn is required to allow authors
%% to have multiple separate bibliographies (here: one for main text and one for SI).
%%
%% For some reasons it does seem to be necessary to reload the bibpackage here since
%% ---at least for my system---adding the following to the YAML header instead does
%% not seem to work:
%%
%%    biblio-style: apa
%%    biblatexoptions: [backend=biber,maxbibnames=999,style=apa]
%% 
%% curiously removing the following from the YAML header:
%%
%%     citation_package: biblatex
%%
%% makes knitr (or pandoc?) revert back to the default bibtex treatment, which does 
%% not allow multiple bibliographies.
% \usepackage[backend=biber, maxbibnames=999, style=apa]{biblatex}
% \addbibresource{latex-stuff/library.bib}
% \newcommand{\brefsection}{\begin{refsection}}
% \newcommand{\erefsection}{\end{refsection}}

\newcommand{\changelocaltocdepth}[1]{%
  \addtocontents{toc}{\protect\setcounter{tocdepth}{#1}}%
  \setcounter{tocdepth}{#1}%
}
\setcounter{tocdepth}{-10}
\changelocaltocdepth{-10}
